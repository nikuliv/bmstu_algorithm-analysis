\chapter{Аналитическая часть}
\newcommand\tab[1][1cm]{\hspace*{#1}}

В данном разделе рассматриваются различные методы нахождения расстояния Левенштейна (матричный, рекурсивный, рекурсивный с использованием кэширования), рекурсивный способ поиска расстояния Дамерау-Левенштейна.

\section{Определение}
Расстояние Левенштейна \cite{Leshenko} между двумя строками — это минимальное количество операций вставки, удаления и замены, необходимых для превращения одной строки в другую.

Цены операций могут зависеть от вида операции (вставка, удаление, замена) и/или от участвующих в ней символов, отражая разную вероятность разных ошибок при вводе текста, и т. п. В общем случае:
\begin{itemize}
	\item $w(a,b)$ — цена замены символа $a$ на символ $b$.
	\item $w(\lambda,b)$ — цена вставки символа $b$.
	\item $w(a,\lambda)$ — цена удаления символа $a$.
\end{itemize}

Для решения задачи о редакционном расстоянии необходимо найти последовательность замен, минимизирующую суммарную цену. Расстояние Левенштейна является частным случаем этой задачи при
\begin{itemize}
	\item $w(a,a)=0$.
	\item $w(a,b)=1, \medspace a \neq b$.
	\item $w(\lambda,b)=1$.
	\item $w(a,\lambda)=1$.
\end{itemize}

\clearpage

\section{Рекурсивный алгоритм нахождения расстояния Левенштейна}

Расстояние Левенштейна между двумя строками a и b может быть вычислено по формуле \ref{rec_lev}, где $|a|$ означает длину строки $a$; $a[i]$ — i-ый символ строки $a$ , функция $D(i, j)$ определена как:
\begin{equation}
	\label{rec_lev}
	D(s1[1..i], s2[1..j]) = 
	\begin{cases}
		0,~if~i == 0,~j == 0; \\
		i,~if~i > 0,~j == 0; \\
		j,~if~i == 0, j > 0; \\
		min
		\begin{cases}
			~D(s1[1..i],~s2[1..j-1) + 1, \\
			~D(s1[1..i-1],~s2[1..j]) + 1, \\
			~D(s1[1..i-1],~s2[1..j-1]) + 
			\left[
			\begin{gathered}
				1,~if~s1[i] == s2[j], \\
				0,~else \\
			\end{gathered}
			\right.
		\end{cases}
	\end{cases}	
\end{equation}


Рекурсивный алгоритм реализует формулу \ref{rec_lev}.
Функция $D$ составлена из следующих соображений:
\begin{enumerate}[label={\arabic*)}]
	\item для перевода из пустой строки в пустую требуется ноль операций;
	\item для перевода из пустой строки в строку $a$ требуется $|a|$ операций;
	\item для перевода из строки $a$ в пустую требуется $|a|$ операций;
\end{enumerate}
Для перевода из строки $a$ в строку $b$ требуется выполнить последовательно некоторое количество операций (удаление, вставка, замена) в некоторой последовательности. Последовательность проведения любых двух операций можно поменять, порядок проведения операций не имеет никакого значения. Полагая, что $a', b'$  — строки $a$ и $b$ без последнего символа соответственно, цена преобразования из строки $a$ в строку $b$ может быть выражена как:
	\begin{enumerate}[label={\arabic*)}]
		\item сумма цены преобразования строки $a$ в $b$ и цены проведения операции удаления, которая необходима для преобразования $a'$ в $a$;
		\item сумма цены преобразования строки $a$ в $b$  и цены проведения операции вставки, которая необходима для преобразования $b'$ в $b$;
		\item сумма цены преобразования из $a'$ в $b'$ и операции замены, предполагая, что $a$ и $b$ оканчиваются разные символы;
		\item цена преобразования из $a'$ в $b'$, предполагая, что $a$ и $b$ оканчиваются на один и тот же символ.
	\end{enumerate}
Минимальной ценой преобразования будет минимальное значение приведенных вариантов.

Существенным недостатком данного алгоритма является его сложность, которая имеет экспоненциальную зависимость, при этом параметры в вызываемых функциях могут повторяться, то есть будут пересчитываться уже известные значения. 
\section{Матричный алгоритм нахождения расстояния Левенштейна}

Прямая реализация формулы \ref{rec_lev} может быть малоэффективна по времени исполнения при больших $i, j$, т. к. множество промежуточных значений $D(i, j)$ вычисляются заново множество раз подряд. Для оптимизации нахождения расстояния Левенштейна можно использовать матрицу в целях хранения соответствующих промежуточных значений \ref{matrix_lev}. В таком случае алгоритм представляет собой построчное заполнение матрицы
$A_{|a|,|b|}$ значениями $D(i, j)$.

\begin{equation}
	\label{matrix_lev}
	D_{i,j} = min
	\begin{cases}
		(D)~D_{i-1,j} + 1, \\
		(I)~D_{i,j-1} + 1, \\
		(R)~D_{i-1,j-1} + 
		\left[
		\begin{gathered}
			1,~if~s1[i] == s2[j]; \\
			0,~else \\
		\end{gathered}
		\right.
	\end{cases}
\end{equation}

Данный алгоритм позволяет уменьшить временные затраты на вычисления за счёт хранения промежуточных значений. Основным недостатком является большой объём затрачиваемой под хранение матрицы памяти.

\section{Рекурсивный алгоритм нахождения расстояния Левенштейна с кэшированием}
\label{sec:recmat}

Рекурсивный алгоритм заполнения можно оптимизировать по времени выполнения с использованием кэширования. Кэширование - это высокоскоростной уровень хранения, на котором требуемый набор данных временного характера. \cite{IBM} Благодаря наличию кэша, можно будет подставлять в формулу уже вычисленное ранее значение, если такое имеется.

\section{Расстояния Дамерау — Левенштейна}

Расстояние Дамерау — Левенштейна может быть найдено по формуле \ref{eq:d}, которая задана как
\begin{equation}
	\label{eq:d}
	d_{a,b}(i, j) = \begin{cases}
		\max(i, j), &\text{if }\min(i, j) = 0\\
		\min = \begin{cases} 
			\qquad d_{a,b}(i-1, j) + 1,\\
			\qquad d_{a,b}(i, j-1) + 1,\\
			\qquad d_{a,b}(i-1, j-1) + m(a,b),\\
			\qquad d_{a,b}(i-2, j-2) + 1,\\
		\end{cases} &\text{if } \begin{cases}i,j > 1;\\a_{i} = b_{j-1}\\ a_{i-1} = b_{j}\end{cases} \\
		\min = \begin{cases} 
			\qquad d_{a,b}(i-1, j) + 1,\\
			\qquad d_{a,b}(i, j-1) + 1,\\
			\qquad d_{a,b}(i-1, j-1) + m(a,b),\\
		\end{cases} &\text{else }
		\end{cases}.
\end{equation}

Формула выводится по тем же соображениям, что и формула (\ref{eq:d}).
Как и в случае с рекурсивным методом, применение этой формулы неэффективно по времени исполнения.

\section*{Вывод}

Входными данными являются две строки в любой раскладке. 
На выход подаётся целое число - расстояние Левенштейна или Демерау-Левенштейна.


Программа должна предоставлять доступ к возможностям:
\begin{itemize}
	\item нахождение расстояния Левенштейна рекурсивно;
	\item нахождение расстояния Левенштейна матрично;
	\item нахождение расстояния Левенштейна рекурсивно с использованием кэширования;
	\item нахождение расстояния Дамерау-Левенштейна рекурсивно;
\end{itemize}


К программе предъявляются требования:
\begin{itemize}
	\item Выполнение вычислений не должно занимать более 1 секунды (при строках с длиной <= 9).
\end{itemize}